\documentclass{wg21}

\usepackage{xcolor}
\usepackage{soul}
\usepackage{ulem}
\usepackage{fullpage}
\usepackage{parskip}
\usepackage{csquotes}
\usepackage{listings}
\usepackage{minted}
\usepackage{enumitem}

\lstdefinestyle{base}{
  language=c++,
  breaklines=false,
  basicstyle=\ttfamily\color{black},
  moredelim=**[is][\color{green!50!black}]{@}{@},
  escapeinside={(*@}{@*)}
}

\newcommand{\cc}[1]{\mintinline{c++}{#1}}
\newminted[cpp]{c++}{}

\title{Extending acceptable predicates and functions in standard library algorithms}
\docnumber{P0XXXR0}
\audience{LEWG}
\author{Georg Rudoy}{georgii.rudoi@phystech.edu}

\begin{document}
\maketitle

\section{Revision history}
\begin{itemize}
  \item R0 --- Initial draft
\end{itemize}

\section{Abstract}

This paper proposes a (strictly relaxing) change in the definitions of several
algorithms in the standard library: \cc{std::any_of}, \cc{std::find_if} and the
likes are currently defined in terms of function call syntax on the predicate/function,
forbidding passing pointers to members where they otherwise would be sufficient.

\section{Motivation}

\begin{cpp}
  struct Standard
  {
      bool isCurrent;
  };

  std::vector<Standard> stds;

  auto hasCurrent = std::any_of(stds.begin(), stds.end(),
          &Standard::isCurrent);                                // nope
  auto hasCurrent = std::any_of(stds.begin(), stds.end(),
          [](auto&& std) { return std.isCurrent; });            // ok 
\end{cpp}

\begin{cpp}
  class Employee
  {
  public:
      bool shouldBeFired() const;
      void fire();
  };

  std::vector<Employee> emps;

  auto poorFellow = std::find_if(emps.begin(), emps.end(),
          &Employee::shouldBeFired);                            // nope
  auto poorFellow = std::find_if(emps.begin(), emps.end(),
          [](auto&& emp) { return emp.shouldBeFired(); });      // ok

  std::for_each(emps.begin(), emps.end(), &Employee::fire);     // nope
  std::for_each(emps.begin(), emps.end(),
          [](auto&& emp) { emp.fire(); });                      // ok
\end{cpp}

\begin{cpp}
  class Employee
  {
  public:
      bool knowsLanguage(const Language& lang) const;
  };

  std::vector<Employee> emps;
  std::vector<Language> langs;

  auto translator = std::find_first_of(emps.begin(), emps.end(),
          langs.begin(), langs.end(),
          &Employee::knowsLanguage);                            // nope
  auto translator = std::find_first_of(emps.begin(), emps.end(),
          langs.begin(), langs.end(),
          [](auto&& emp, auto&& lang) { return emp.knowsLanguage(lang); });
                                                                // ok
\end{cpp}

\section{The fix}

Replace \cc{pred(*i)} and the likes with \cc{std::invoke(pred, *i)}
and the likes.

Clarify the passed function object should be applied using \tcode{std::invoke}
where relevant.

\section{Proposed wording}

In \textbf{[alg.all_of]/2}:
\begin{quote}
  \emph{Returns:} \tcode{true} if \tcode{[first, last)} is empty or if
  \removed{\tcode{pred(*i)}} \added{\tcode{std::invoke(pred, *i)}} is
  \tcode{true} for every iterator \tcode{i} in the range \tcode{[first, last)},
  and \tcode{false} otherwise.
\end{quote}

In \textbf{[alg.any_of]/2}: ditto.

In \textbf{[alg.none_of]/2}: ditto.

In \textbf{[alg.find]/2}: ditto.

In \textbf{[alg.count]/1}: ditto.

In \textbf{[alg.foreach]/2}:
\begin{quote}
  \emph{Effects:} Applies \tcode{f} \added{using \tcode{std::invoke}} to the
  result of dereferencing every iterator in the range \tcode{[first, last)},
  starting from \tcode{first} and proceeding to \tcode{last - 1}.
  [ \emph{Note:} If the type of \tcode{first} satisfies the requirements of a
  mutable iterator, \tcode{f} may apply non-constant functions through the
  dereferenced iterator. --- \emph{end note} ]
\end{quote}

In \textbf{[alg.foreach]/7}: ditto.

In \textbf{[alg.foreach]/13}: ditto.

In \textbf{[alg.foreach]/18}: ditto.

In \textbf{[alg.find.end]/2}:
\begin{quote}
  \emph{Returns}: The last iterator \tcode{i} in the range
  \tcode{[first1, last1 - (last2 - first2))} such that for every non-negative
  integer \tcode{n < (last2 - first2)}, the following corresponding conditions
  hold: \tcode{*(i + n) == *(first2 + n)},
  \removed{\tcode{pred(*(i + n), *(first2 + n)) != false}}
  \added{\tcode{std::invoke(pred, *(i + n), *(first2 + n)) != false}}.
  Returns \tcode{last1} if \tcode{[first2, last2)} is empty or if no such iterator is found.
\end{quote}

In \textbf{[alg.find.first.of]/2}: ditto.

In \textbf{[alg.adjacent.find]/1}: ditto.

In \textbf{[alg.mismatch]/2}: ditto.

In \textbf{[alg.equal]/2}: ditto.

In \textbf{[alg.is_permutation]/3}: ditto.

In \textbf{[alg.search]/2}: ditto.

\end{document}
